\documentclass[a4paper,12pt,ngerman]{article}

\usepackage[ngerman]{babel}
\usepackage[utf8]{inputenc}
%\usepackage{listings}
%\lstset{basicstyle=\ttfamily\small}
%\usepackage{verbatim}
%\usepackage{dingbat}
\usepackage{graphics}
\usepackage[dvipsnames,usenames]{color}
\usepackage{amsmath,amssymb}
\usepackage{uebung}
\usepackage{../../Uebung/asp}
\usepackage{url}
\usepackage{hyperref}
\usepackage{tikz}

\sloppy
\DeclareFontFamily{U}{bulb}{}
\DeclareFontShape{U}{bulb}{m}{n}{<-> lightbulb10}{}
\newcommand{\lightbulb}{{\usefont{U}{bulb}{m}{n}A}}

\newcommand*\circled[1]{\hspace{0.6mm}\tikz[baseline=(char.base)]{
    \node[shape=circle,draw,fill,inner sep=0.2mm] (char) {\color{White}\footnotesize{\textsf{\textbf{#1}}}};}}

\begin{document}
\begin{PraktikumsAufgabe}{1}{(Hop, Skip and Jump)}
\vspace*{-4mm}
\paragraph{Problembeschreibung.}
%
%
In dieser Praktikumsaufgabe sollen Sie ein Logikpuzzle mit einem ASP-Encoding lösen.
Das Ziel des Puzzles besteht darin, einen Pfad zwischen vordefinierten Start- und Endpunkten zu finden.
Der Pfad besteht dabei aus unterschiedlich langen Sprüngen in horizontaler oder vertikaler Richtung,
deren Länge dem Muster 1, 2, 3, 1, 2, 3, \dots\ folgt.
Ihre Aufgabe besteht darin, die Zellen zu bestimmen, in denen zwischen den Sprüngen gelandet wird.
Einige dieser Zellen sind bereits vorgegeben, jedoch ohne Reihenfolge.
%
\begin{figure}[h]
\begin{center}
\begin{tabular}{c@{~}|@{}p{5mm}@{}|@{}p{5mm}@{}|@{}p{5mm}@{}|@{}p{5mm}@{}|@{}p{5mm}@{}|@{}p{5mm}@{}|l}
\cline{2-6}
\textsf{1} & & & & & \hspace{.8mm}{G}
\\\cline{2-6}
\textsf{2} & & ~$\bullet$ & & &
\\\cline{2-6}
\textsf{3} & & & & &
\\\cline{2-6}
\textsf{4} & ~S & & & &
\\\cline{2-6}
\multicolumn{1}{@{}c@{}}{} &
\multicolumn{1}{@{}c@{}}{\textsf{1}} & \multicolumn{1}{@{}c@{}}{\textsf{2}} & \multicolumn{1}{@{}c@{}}{\textsf{3}} & \multicolumn{1}{@{}c@{}}{\textsf{4}} & \multicolumn{1}{@{}c@{}}{\textsf{5}} 
\end{tabular}
\hspace{2cm}
\begin{tabular}{c@{~}|@{}p{5mm}@{}|@{}p{5mm}@{}|@{}p{5mm}@{}|@{}p{5mm}@{}|@{}p{5mm}@{}|@{}p{5mm}@{}|l}
\cline{2-6}
\textsf{1} & & & & & \circled{4}
\\\cline{2-6}
\textsf{2} & & \circled{2} & & & \circled{3}
\\\cline{2-6}
\textsf{3} & & & & &
\\\cline{2-6}
\textsf{4} & \circled{0} & \circled{1} & & &
\\\cline{2-6}
\multicolumn{1}{@{}c@{}}{} &
\multicolumn{1}{@{}c@{}}{\textsf{1}} & \multicolumn{1}{@{}c@{}}{\textsf{2}} & \multicolumn{1}{@{}c@{}}{\textsf{3}} & \multicolumn{1}{@{}c@{}}{\textsf{4}} & \multicolumn{1}{@{}c@{}}{\textsf{5}} 
\end{tabular}


\end{center}
\vspace*{-6mm}
\caption{Spielfeld der Gr\"o{\ss}e $5\times 4$ (links) und 
         besuchte Zellen (rechts).\label{fig:laby}}
\end{figure}

\noindent
Ein Beispiel ist in Abbildung~\ref{fig:laby} dargestellt.
Die linke Seite zeigt das Spielfeld mit Start (S) und Ziel (G) sowie einem vorgegebenen Zwischenhalt in der Zelle (2,2).
Auf der rechten Seite ist die (in diesem Fall eindeutige) Lösung inklusive der Reihenfolge der besuchten Zellen zu sehen.
Wie vorgegeben ist die Distanz zwischen der Startzelle und dem ersten Zwischenhalt 1, zwischen dem ersten und zweiten Zwischenhalt 2, usw.

%\newpage
\paragraph{Repräsentation in ASP.}
Die Seitenlängen des Gitters,
die vorgebenen Start- und Zielpunkte und
die vorgegebenen Zwischenhalte stellen wir wie folgt durch Fakten % folgender Form 
dar:
%
% der folgenden Form dar:%
% \newpage
\vspace{-1.5mm}
\begin{verbatim}
 cols(C).       % Number of columns
 rows(R).       % Number of rows
 numsteps(N).   % Number of steps necessary
 start(X,Y).    % Start cell
 goal(X,Y).     % Goal cell
 dot(X,Y).      % Predefined intermediate step
\end{verbatim}
\vspace{-1mm}
Z.B.\ wird das in Abbildung~\ref{fig:laby} (links) dargestellte Feld % Spielfeld
durch folgende Fakten beschrieben:%
\vspace{-1.5mm}
\begin{verbatim}
 cols(5).  rows(4).  numsteps(4).  start(1,4).  goal(5,1).  dot(2,2).
\end{verbatim}
\vspace{-1mm}
Eine Lösung, d.h.\ eine Liste von Zellen, die besucht werden, wird durch Atome folgender Form repräsentiert:
\vspace{-1.5mm}
\begin{verbatim}
 step(X0,Y0,0)  step(X1,Y1,1)  ...  step(Xn,Yn,n)  % Visited cells
\end{verbatim}
\vspace{-1mm}
Die in Abbildung~\ref{fig:laby} (rechts) dargestellte Lösung wird z.B.\
durch die folgenden Atome innerhalb einer Antwortmenge beschrieben:
\vspace{-1.5mm}
\begin{verbatim}
 step(1,4,0)  step(2,4,1)  step(2,2,2)  step(5,2,3)  step(5,1,4)
\end{verbatim}
\vspace{-1mm}
Beachten Sie, dass die Anzahl der Schritte den Startpunkt, der die Nummer 0 hat,
nicht einschließt. % und dieser bekommt.
% Sowohl 
Die Start- % als auch 
und die Zielzelle müssen jedoch in der Lösung enthalten sein.

\paragraph{Framework.}

\noindent
In dem Archiv 
\href{http://www.cs.uni-potsdam.de/wv/lehre/12WS/12-Antwortprog/hop.tar.gz}{\texttt{hop.tar.gz}}
finden Sie die Datei \texttt{hop.lp} und sieben Beispielinstanzen.
Die Datei \texttt{hop.lp} ist von Ihnen mit Ihrem ASP-Encoding zu ergänzen.
Wenn Sie Ihr Encoding einreichen,
müssen die folgenden Zeilen in \texttt{hop.lp} enthalten sein:%
\footnote{Diese Anweisungen sorgen dafür, dass alle Atome, die nicht zur Repräsentation von Lösungen gehören, bei der Ausgabe von Antwortmengen ausgeblendet werden.}%
\vspace{-1.5mm}
\begin{verbatim}
 #hide.
 #show step/3.
\end{verbatim}
% \begin{tabular}{l}
% \code{\#hide.}\\
% \code{\#show light/2.}\\
% \end{tabular}
\vspace{-1mm}
% \noindent
{\small\sf Sie m\"ussen Ihr Encoding in einer Datei mit dem Namen \texttt{hop.lp}
           einreichen.
           Neben \code{step/3} d\"urfen in der eingereichten Version keine weiteren
           Pr\"adikate in der Ausgabe eingeblendet sein!}

\noindent


\noindent

\paragraph{Formalitäten.}
%
Sie können die Praktikumsaufgabe in Gruppen von \textbf{bis zu zwei} Studenten gemeinsam bearbeiten.
Verschiedene Gruppen müssen verschiedene Lösungen einreichen.
Bei Plagiaten wird die Praktikumsaufgabe für alle beteiligten Gruppen als ``nicht bestanden'' gewertet.
Reichen Sie Ihr Encoding bitte bis zum \textbf{30.11.3012} % TODO
über % die von der Moodle-Seite zur Vorlesung aus verlinkte 
\href{http://yeti.haiti.cs.uni-potsdam.de/student/showPractical.do?practicalId=36}{YETI} % -Plattform
ein.
(Alle Gruppenmitglieder müssen bei YETI einen Account haben und als 
 Gruppenmitglieder registriert sein!)
Achten Sie darauf, dass Sie Ihr Encoding in einer Datei mit dem Namen \code{hop.lp} einreichen,
wobei der Dateiname ausschließlich Kleinbuchstaben enthält.

\noindent
Neben den sieben vorgegebenen Instanzen (von denen die ersten fünf jeweils eine eindeutige L\"osung
und die sechste und siebte keine bzw. zwei Lösungen haben)
testen wir Ihr Encoding auf weiteren Ihnen unbekannten Instanzen.
Auch auf diesen weiteren Instanzen muss Ihr Encoding \emph{korrekt} funktionieren,
damit die Praktikumsaufgabe als bestanden gilt.
Wenn Sie Ihr Encoding bei YETI hochladen, wird es dort automatisch getestet
(mit geringer zeitlicher Verz\"ogerung).
Falls dabei Fehler auftreten, k\"onnen Sie dies %e
den Statusmeldungen von YETI entnehmen.
Bitte korrigieren Sie eventuelle Fehler umgehend selbständig oder
(bei Schwierigkeiten) in Rücksprache mit uns.
% Zur Analyse der resultierenden Effizienz von Encodings werden wir ein Vergleichsencoding 
% bei YETI hochladen, das Ihnen einen Anhalts\-punkt bzgl.\ der Performanz Ihres
% eigenen Encodings gibt.


\paragraph{Empfehlungen und Hinweise:}
\begin{itemize}
\item Kommandozeilenaufrufe zum Finden aller Antwortmengen haben die folgende Form:%
\vspace{-1.5mm}%
\begin{verbatim}
 $  gringo hop.lp levelX.lp | clasp 0
\end{verbatim}
\vspace{-1mm}
% Überlegen Sie sich, welche Bedingungen Sie zur Charakterisierung von L\"osungen
%       testen wollen und definieren Sie geeignete Pr\"adikate, die diese Tests erm\"oglichen.
% %\item Verwenden Sie die Kommandozeilenoption \texttt{--solution-recording} zur
% %      Effizienz\-steigerung bei der Berechnung aller L\"osungen, z.B.\ wie in dem folgenden Aufruf:
%
% \begin{tabular}{@{}l}%
% \texttt{[user@local \textasciitilde] clingo lights.lp test01.lp --solution-recording 0}
% \end{tabular}      
\item Wenn Sie an einer Stelle nicht weiterkommen, können Sie sich gern an uns wenden.
      Wir versuchen alle Fragen bestmöglich zu beantworten.
      Fragen und Bemerkungen k\"onnen Sie pers\"onlich an uns richten,
      in \href{http://moodle.cs.uni-potsdam.de/course/view.php?id=39}{Moodle} (Forum/Wiki) stellen oder per Mail an
      \href{mailto:asp1@cs.uni-potsdam.de}{\texttt{asp1@cs.uni-potsdam.de}} senden.
\item Fangen Sie bald mit der Bearbeitung der Aufgabe an, damit Ihnen die Zeit nicht davonläuft.
      (Sollten Sie trotzdem Schwierigkeiten mit der Einhaltung des Termins
       haben, dann wenden Sie sich bitte an uns, anstatt eine beliebige Lösung
       zu kopieren!)
% \item Behalten Sie Ihren Quelltext für sich.
%       Sie helfen Ihren Kommilitonen nicht, wenn Sie Ihren Quelltext weitergeben, da das Nachvollziehen eines fremden Encodings schwieriger ist als ein eigenes zu entwerfen.
\end{itemize}

\end{PraktikumsAufgabe}
\end{document}
