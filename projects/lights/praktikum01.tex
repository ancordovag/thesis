\documentclass[a4paper,12pt,ngerman]{article}

\usepackage[ngerman]{babel}
\usepackage[utf8]{inputenc}
%\usepackage{listings}
%\lstset{basicstyle=\ttfamily\small}
%\usepackage{verbatim}
%\usepackage{dingbat}
\usepackage{graphics}
\usepackage[dvipsnames,usenames]{color}
\usepackage{amsmath,amssymb}
\usepackage{exercises}
\usepackage{../../../asp}
\usepackage{url}
\usepackage{hyperref}

\sloppy
\DeclareFontFamily{U}{bulb}{}
\DeclareFontShape{U}{bulb}{m}{n}{<-> lightbulb10}{}
\newcommand{\lightbulb}{{\usefont{U}{bulb}{m}{n}A}}

\begin{document}
\begin{PraktikumsAufgabe}{1}{(Lights)}
\vspace*{-4mm}
\paragraph{Problembeschreibung.}
%
%
In dieser Praktikumsaufgabe sollen Sie ein Logikpuzzle,
bei dem ein zweidimensionales Spielfeld mit Lichtern ausgeleuchtet werden soll,
mit einem ASP-Encoding l\"osen.
Das Spielfeld ist ein quadratisches Gitter, 
wobei einige Zellen vorgegeben sind.
Die vorgegebenen Zellen sind entweder leer oder enthalten 
eine ganze Zahl zwischen $0$ und $4$.
Das Ziel des Puzzles besteht darin,
Lichter so auf den nicht vorgegebenen Zellen zu platzieren,
dass jede nicht vorgegebene Zelle von (mindestens) einem Licht
horizontal oder vertikal beleuchtet wird,
wobei
Lichtstrahlen nicht durch vorgegebene (leere
oder eine Zahl enthaltende) Zellen hindurchleuchten.
Zus\"atzliche Randbedingungen bestehen darin,
dass sich keine zwei Lichter gegenseitig beleuchten d\"urfen
und dass f\"ur jede Zelle mit einer Zahl~$n$ genau $n$ horizontal
oder vertikal angrenzende Zellen jeweils ein Licht enthalten m\"ussen.
%
\begin{figure}[h]
\begin{center}
%\vspace*{-2mm}
\begin{tabular}{c@{~}|@{}p{5mm}@{}|@{}p{5mm}@{}|@{}p{5mm}@{}|@{}p{5mm}@{}|@{}p{5mm}@{}|@{}p{5mm}@{}|@{}p{5mm}@{}|l}
% & \multicolumn{1}{@{}c@{}}{} & \multicolumn{1}{@{}c@{}}{\raisebox{-2pt}{4}}
% \\
\cline{2-7}
\textsf{1} & & \hspace*{5.1mm}{\color{Tan}\rule[-1.6mm]{5mm}{5mm}} & ~$1$ & \hspace*{0.0mm}{\color{Tan}\rule[-1.6mm]{5mm}{5mm}} & & 
\\\cline{2-7}
\textsf{2} & & & & & &
\\\cline{2-7}
\textsf{3} & \hspace*{5.1mm}{\color{Tan}\rule[-1.6mm]{5mm}{5mm}} & ~$0 $ & & & \hspace*{0.0mm}{\color{Tan}\rule[-1.6mm]{5mm}{5mm}} & 
\\\cline{2-7}
\textsf{4} & & \hspace*{0.0mm}{\color{Tan}\rule[-1.6mm]{5mm}{5mm}} & \hspace*{10.1mm}{\color{Tan}\rule[-1.6mm]{5mm}{5mm}} & & ~$2$ & 
\\\cline{2-7}
\textsf{5} & & & & & &
\\\cline{2-7}
\textsf{6} & \hspace*{10.1mm}{\color{Tan}\rule[-1.6mm]{5mm}{5mm}} & \hspace*{10.1mm}{\color{Tan}\rule[-1.6mm]{5mm}{5mm}} & ~$0$ & ~$1$ & & 
\\\cline{2-7}
% & \multicolumn{1}{@{}c@{}}{} & \multicolumn{1}{@{}c@{}}{} & \multicolumn{1}{@{}c@{}}{2} & \multicolumn{1}{@{}c@{}}{2} & \multicolumn{1}{@{}c@{}}{}
% \\[-2pt]
\multicolumn{1}{@{}c@{}}{} &
\multicolumn{1}{@{}c@{}}{\textsf{1}} & \multicolumn{1}{@{}c@{}}{\textsf{2}} & \multicolumn{1}{@{}c@{}}{\textsf{3}} & \multicolumn{1}{@{}c@{}}{\textsf{4}} & \multicolumn{1}{@{}c@{}}{\textsf{5}} & \multicolumn{1}{@{}c@{}}{\textsf{6}} 
\end{tabular}
\hspace{2cm}
\begin{tabular}{c@{~}|@{}p{5mm}@{}|@{}p{5mm}@{}|@{}p{5mm}@{}|@{}p{5mm}@{}|@{}p{5mm}@{}|@{}p{5mm}@{}|@{}p{5mm}@{}|l}
% & \multicolumn{1}{@{}c@{}}{} & \multicolumn{1}{@{}c@{}}{\raisebox{-2pt}{4}}
% \\
\cline{2-7}
\textsf{1} & ~\lightbulb & \hspace*{5.1mm}{\color{Tan}\rule[-1.6mm]{5mm}{5mm}} & ~$1$ & \hspace*{0.0mm}{\color{Tan}\rule[-1.6mm]{5mm}{5mm}} & ~\lightbulb & 
\\\cline{2-7}
\textsf{2} & & & ~\lightbulb & & &
\\\cline{2-7}
\textsf{3} & \hspace*{5.1mm}{\color{Tan}\rule[-1.6mm]{5mm}{5mm}} & ~$0 $ & & & \hspace*{0.0mm}{\color{Tan}\rule[-1.6mm]{5mm}{5mm}} & 
\\\cline{2-7}
\textsf{4} & & \hspace*{0.0mm}{\color{Tan}\rule[-1.6mm]{5mm}{5mm}} & \hspace*{10.1mm}{\color{Tan}\rule[-1.6mm]{5mm}{5mm}} & ~\lightbulb & ~$2$ & ~\lightbulb
\\\cline{2-7}
\textsf{5} & & ~\lightbulb & & & &
\\\cline{2-7}
\textsf{6} & \hspace*{10.1mm}{\color{Tan}\rule[-1.6mm]{5mm}{5mm}} & \hspace*{10.1mm}{\color{Tan}\rule[-1.6mm]{5mm}{5mm}} & ~$0$ & ~$1$ & ~\lightbulb & 
\\\cline{2-7}
% & \multicolumn{1}{@{}c@{}}{} & \multicolumn{1}{@{}c@{}}{} & \multicolumn{1}{@{}c@{}}{2} & \multicolumn{1}{@{}c@{}}{2} & \multicolumn{1}{@{}c@{}}{}
% \\[-2pt]
\multicolumn{1}{@{}c@{}}{} &
\multicolumn{1}{@{}c@{}}{\textsf{1}} & \multicolumn{1}{@{}c@{}}{\textsf{2}} & \multicolumn{1}{@{}c@{}}{\textsf{3}} & \multicolumn{1}{@{}c@{}}{\textsf{4}} & \multicolumn{1}{@{}c@{}}{\textsf{5}} & \multicolumn{1}{@{}c@{}}{\textsf{6}} 
\end{tabular}
\end{center}
\vspace*{-6mm}
\caption{Spielfeld der Gr\"o{\ss}e $6$ (links) und 
         Platzierung von Lichtern (rechts).\label{fig:laby}}
\end{figure}

\noindent
Ein Beispiel ist in Abbildung~\ref{fig:laby} dargestellt.
Die linke Seite zeigt das Spielfeld mit acht vorgegebenen Zellen,
von denen drei leer sind und f\"unf jeweils eine Zahl enthalten.
Auf der rechten Seite sind zus\"atzlich die Lichter dargestellt,
deren Platzierung eine (in diesem Fall eindeutige) L\"osung f\"ur
das Spielfeld bildet.
Beobachten Sie, dass jede nicht vorgegebene (wei{\ss}e) Zelle
entweder ein Licht enth\"alt oder eine horizontale oder vertikale
Verbindung zu einer Zelle mit einem Licht hat.
Zwischen den Zellen mit Lichtern gibt es dagegen keine solchen
Verbindungen, z.B.\ sind die Zellen $($\textsf{4}$,$\textsf{4}$)$
und $($\textsf{4}$,$\textsf{6}$)$ durch die Zelle mit der Zahl~$2$
voneinander getrennt.
Nicht zuletzt hat jede vorgegebene Zelle mit einer Zahl die geforderte
Anzahl an horizontal oder vertikal angrenzenden Zellen mit Lichtern.

\noindent
Das beschriebene Puzzle kann auch online ausprobiert werden.
Das Probespielen einiger Levels ist wahrscheinlich die beste Methode,
sich die Regeln zu vergegenwärtigen und L\"osungsideen zu entwickeln.
Das Spiel (als Java-Applet) finden Sie hier:

\url{http://www.janko.at/Raetsel/Akari/}

%\newpage
\paragraph{Repräsentation in ASP.}
Die Seitenlängen des Gitters,
die vorgegebenen Zellen mit Zahlen und
die vorgegebenen leeren Zellen stellen wir 
durch Fakten folgender Form dar:
% der folgenden Form dar:%
% \newpage
\vspace{-1mm}
\begin{verbatim}
 rows(R).       % Number of rows
 cols(C).       % Number of columns
 digit(X,Y,D).  % Digit D at position (X,Y), i.e., in row X and column Y
 empty(X,Y).    % Empty cell at position (X,Y), i.e., in row X and column Y
\end{verbatim}

\noindent
Z.B.\ wird das in Abbildung~\ref{fig:laby} (links) dargestellte Spielfeld % wird z.B.\
durch folgende Fakten beschrieben:
\vspace{-1mm}
\begin{verbatim}
 rows(6).       cols(6).
 digit(1,3,1).  digit(3,2,0).  digit(4,5,2).  digit(6,3,0).  digit(6,4,1).
 empty(1,4).    empty(3,5).    empty(4,2).
\end{verbatim}
%\vspace{-1mm}
Eine Lösung, d.h.\ eine Platzierung von Lichtern auf nicht vorgegebenen
Zellen, die allen oben beschriebenen Bedingungen gen\"ugt,
wird durch Atome folgender Form repräsentiert:%
\vspace{-1mm}
\begin{verbatim}
 light(X1,Y1)  light(X2,Y2)  ...  light(Xn,Yn)  % Cells with lights
\end{verbatim}
%\vspace{-1mm}
Die in Abbildung~\ref{fig:laby} (rechts) dargestellte Lösung wird z.B.\
durch die folgenden Atome innerhalb einer Antwortmenge beschrieben:
\vspace{-1mm}
\begin{verbatim}
 light(1,1)  light(1,5)  light(2,3)  light(4,4) 
 light(4,6)  light(5,2)  light(6,5) 
\end{verbatim}
%\vspace{-1mm}

\paragraph{Framework.}

\noindent
In dem Archiv \texttt{lights.zip} finden Sie die Datei \texttt{lights.lp} und achtzehn Beispielinstanzen.
Die Datei \texttt{lights.lp} ist von Ihnen mit Ihrem ASP-Encoding zu ergänzen.
Wenn Sie Ihr Encoding einreichen,
dann müssen die folgenden Zeilen in \texttt{lights.lp} enthalten sein:
\footnote{Diese Anweisungen sorgen dafür, dass alle Atome, die nicht zur Repräsentation von Lösungen gehören, bei der Ausgabe von Antwortmengen ausgeblendet werden.}
\vspace{-1mm}
\begin{verbatim}
 #hide.
 #show light/2.
\end{verbatim}
% \begin{tabular}{l}
% \code{\#hide.}\\
% \code{\#show light/2.}\\
% \end{tabular}

\noindent
{\small\sf Sie m\"ussen Ihr Encoding in einer Datei mit dem Namen \texttt{lights.lp}
           einreichen.
           Neben \code{light/2} d\"urfen in der eingereichten Version keine weiteren
           Pr\"adikate in der Ausgabe eingeblendet sein!}

\noindent


\noindent

\paragraph{Formalitäten.}
%
Sie können die Praktikumsaufgabe in Gruppen von \textbf{bis zu zwei} Studenten gemeinsam bearbeiten.
Verschiedene Gruppen müssen verschiedene Lösungen einreichen.
Bei Plagiaten wird die Praktikumsaufgabe für alle beteiligten Gruppen als ``nicht bestanden'' gewertet.
Reichen Sie Ihr Encoding bitte bis zum \textbf{11.12.2011}
über die von der Moodle-Seite zur Vorlesung aus verlinkte YETI-Plattform ein.
(Alle Gruppenmitglieder müssen bei YETI einen Account haben und als 
 Gruppenmitglieder registriert sein!)
% Tragen Sie dort alle Ihre Gruppenmitglieder ein.
Achten Sie darauf, dass Sie Ihr Encoding in einer Datei mit dem Namen \code{lights.lp} einreichen,
wobei der Dateiname ausschließlich Kleinbuchstaben enthält.

\noindent
Neben den achtzehn vorgegebenen Instanzen (mit jeweils einer eindeutigen L\"osung)
testen wir Ihr Encoding auf weiteren Ihnen unbekannten Instanzen.
Auch auf diesen weiteren Instanzen muss Ihr Encoding \emph{korrekt} funktionieren,
damit die Praktikumsaufgabe als bestanden gilt.
Wenn Sie Ihr Encoding bei YETI hochladen, wird es dort automatisch getestet
(mit geringer zeitlicher Verz\"ogerung).
Falls dabei Fehler auftreten, k\"onnen Sie diese den Statusmeldungen von YETI entnehmen.
Zur Analyse der resultierenden Effizienz von Encodings werden wir ein Vergleichsencoding 
bei YETI hochladen, das Ihnen einen Anhalts\-punkt bzgl.\ der Performanz Ihres
eigenen Encodings gibt.

% \noindent{\small\sf
% Encodings, die eine \"ahnliche Performanz wie unser Vergleichsencoding erzielen,
% werden mit \textbf{bis zu 2 Bonuspunkten auf die Klausur} honoriert.}


\paragraph{Empfehlungen und Hinweise:}%\vspace{-2mm}
\begin{itemize}
\item Überlegen Sie sich, welche Bedingungen Sie zur Charakterisierung von L\"osungen
      testen wollen und definieren Sie geeignete Pr\"adikate, die diese Tests erm\"oglichen.
% eine allgemeine (geeignete) Repräsentation, um Pfade zu beschreiben, bevor Sie Ihre Idee in einem ASP-Encoding umsetzen.
\item Verwenden Sie die Kommandozeilenoption \texttt{--solution-recording} zur
      Effizienz\-steigerung bei der Berechnung aller L\"osungen, z.B.\ wie in dem folgenden Aufruf:

\begin{tabular}{@{}l}%
\texttt{[user@local \textasciitilde] clingo lights.lp test01.lp --solution-recording 0}
\end{tabular}
\item Wenn Sie an einer Stelle nicht weiterkommen, können Sie sich gern an uns wenden.
      Wir versuchen alle Fragen bestmöglich zu beantworten.
      Fragen k\"onnen Sie entweder pers\"onlich an uns richten oder per Mail an:
      \texttt{asp1@cs.uni-potsdam.de}.
\item Fangen Sie bald mit der Bearbeitung der Aufgabe an, damit Ihnen die Zeit nicht davonläuft.
      (Sollten Sie trotzdem Schwierigkeiten mit der Einhaltung des Termins
       haben, dann wenden Sie sich bitte an uns, anstatt eine beliebige Lösung
       zu kopieren!)
% \item Behalten Sie Ihren Quelltext für sich.
%       Sie helfen Ihren Kommilitonen nicht, wenn Sie Ihren Quelltext weitergeben, da das Nachvollziehen eines fremden Encodings schwieriger ist als ein eigenes zu entwerfen.
\end{itemize}

\end{PraktikumsAufgabe}
\end{document}
