\documentclass[a4paper,12pt,ngerman]{article}

\usepackage[ngerman]{babel}
\usepackage[utf8]{inputenc}
%\usepackage{listings}
%\lstset{basicstyle=\ttfamily\small}
%\usepackage{verbatim}
%\usepackage{dingbat}
\usepackage{graphics}
\usepackage[dvipsnames,usenames]{color}
\usepackage{amsmath,amssymb}
\usepackage{exercises}
\usepackage{../../../asp}
\usepackage{url}
\usepackage{hyperref}

\sloppy
% \DeclareFontFamily{U}{bulb}{}
% \DeclareFontShape{U}{bulb}{m}{n}{<-> lightbulb10}{}
% \newcommand{\lightbulb}{{\usefont{U}{bulb}{m}{n}A}}

\begin{document}
\begin{PraktikumsAufgabe}{2}{(Metro)}
%
\vspace*{-6mm}
\paragraph{Problembeschreibung.}
%
%
Ein aktuelles Ziel in der Telekommunikation ist die Bereitstellung
schneller und zuverlässiger Glasfasernetze. % (in Zeiten von HD-Streams und Clouddiensten).
Da der Netzaufbau bzw.\ -ausbau kostspielig ist,
sowohl bzgl.\ Material- als auch Baukosten,
werden die Verbindungsstrecken bei der Netzplanung möglichst gering gehalten.
Spielräume bei der Planung ergeben sich daraus,
dass die Standorte sogenannter Metroknoten flexibel gewählt werden können,
wohingegen die zu verbindenden lokalen Endknoten feststehen.
Da bei der Netzoptimierung einige Randbedingungen (siehe unten) zu beachten sind
und die Kombinatorik mit der Anzahl und Variabilität möglicher Metroknoten deutlich ansteigt,
soll die Planung mittels Antwortmengenprogrammierung (ASP) realisiert werden.

\begin{figure}[h]
\setlength{\unitlength}{1cm}
\begin{picture}(4,2.4)(-1.5,0.9)\thicklines
\put(1,1){\circle*{0.2}}\put(0.85,0.6){$\scriptstyle{77}$}
\put(1,2){\circle*{0.2}}\put(0.50,1.95){$\scriptstyle{53}$}
\put(2,2){\circle*{0.2}}\put(1.85,1.65){$\scriptstyle{59}$}
\put(4,1){\circle*{0.2}}\put(3.85,0.6){$\scriptstyle{47}$}
\put(1,3){\circle{0.2}}\put(-0.1,2.9){$\scriptstyle({160},3)$}
\put(2,3){\circle{0.2}}\put(1.45,3.2){$\scriptstyle({200},3)$}
\put(3,1){\circle{0.2}}\put(2.55,0.6){$\scriptstyle({130},2)$}
\put(3,2){\circle{0.2}}\put(2.55,2.2){$\scriptstyle({180},3)$}
\put(3,3){\circle{0.2}}\put(2.55,3.2){$\scriptstyle({240},4)$}
\put(4,2){\circle{0.2}}\put(4.20,1.95){$\scriptstyle({240},4)$}
\end{picture}
\hfill
\begin{picture}(4,2.4)(-1.5,0.9)\thicklines
\put(1,1){\circle*{0.2}}\put(0.85,0.6){$\scriptstyle{77}$}
\put(1,2){\circle*{0.2}}\put(0.50,1.95){$\scriptstyle{53}$}
\put(2,2){\circle*{0.2}}\put(1.85,1.65){$\scriptstyle{59}$}
\put(4,1){\circle*{0.2}}\put(3.85,0.6){$\scriptstyle{47}$}
% \put(1,3){\circle{0.2}}\put(-0.1,2.9){$\scriptstyle({160},3)$}
\put(2,3){\circle{0.2}}\put(1.45,3.2){$\scriptstyle({200},3)$}
\put(3,1){\circle{0.2}}\put(2.55,0.6){$\scriptstyle({130},2)$}
\put(3,2){\circle{0.2}}\put(2.55,2.2){$\scriptstyle({180},3)$}
% \put(3,3){\circle{0.2}}\put(2.55,3.2){$\scriptstyle({240},4)$}
% \put(4,2){\circle{0.2}}\put(4.20,1.95){$\scriptstyle({240},4)$}
\put(1,1){\line(1,0){1.9}}\put(1,1){\line(1,2){0.95}}
\put(1,2){\line(1,1){0.915}}\qbezier(1,2)(2,1.2)(2.9,1.95)
\put(2,2){\line(0,1){0.9}}\put(2,2){\line(1,0){0.9}}
\put(4,1){\line(-1,0){0.9}}\put(4,1){\line(-1,1){0.915}}
\end{picture}
\hfill\mbox{~}
\caption{Ein Netzbeispiel (links) und Verbindungen zu drei Metroknoten (rechts).\label{fig:metro}}
\end{figure}

\noindent
Eine beispielhaftes Netzplanungsproblem ist in Abbildung~\ref{fig:metro} (links) dargestellt.
Die vier schwarzen Punkte symbolisieren lokale Endknoten, die jeweils eine bestimmte, an
den            Knoten angegebene, Netzlast erzeugen.
Dagegen stellen die weißen Punkte mögliche Metroknoten dar, die jeweils mit einem Zahlenpaar
aus maximaler Kapazität und der Maximalanzahl verbundener Endknoten assoziiert sind.
Bei der Netzplanung besteht das Problem darin, aus den Möglichkeiten eine vorgegebene Anzahl
von Metroknoten auszuwählen und ihre Verbindungen zu den Endknoten so festzulegen,
dass die folgenden (harten) Bedingungen erfüllt sind:
\vspace{-3mm}
\begin{enumerate}\itemsep-3pt
  \item Um den Ausfall eines Metroknoten kompensieren zu können,
        muss jeder lokale Endknoten mit genau \emph{zwei} Metroknoten verbunden sein.
  \item Die Anzahl der zu einem Metroknoten verbundenen Endknoten darf        die Maximal\-anzahl
        von Verbindungen des Metroknoten nicht übersteigen.
  \item Die Summe von Lasten der zu einem Metroknoten verbundenen Endknoten darf       die Kapazität
        des Metroknoten nicht übersteigen.
  \item Im Falle des Ausfalls von \emph{zwei} beliebigen Metroknoten
        darf   die Anzahl der vom Netz abgetrennten lokalen Endknoten eine 
        vorgegebene Schranke nicht überschreiten.
\end{enumerate}
\vspace{-3mm}
Eine L\"osung mit~$3$ Metroknoten, die diese      Bedingungen f\"ur die Schranke~$2$ von 
bei einem Doppelausfall
abgetrennten   Endknoten erf\"ullt, ist in Abbildung~\ref{fig:metro} (rechts) dargestellt.

\noindent
Wie oben bereits angesprochen, sind die Verbindungsstrecken zwischen lokalen End- und
Metroknoten m\"oglichst gering zu halten,
wobei die quadratische 
          Euklidische Distanz $$      (x_1{-}x_2)^2+(y_1{-}y_2)^2 $$ als Maß f\"ur
die Strecke zwischen zwei Knoten mit den Koordinaten $(x_1,y_1)$ und $(x_2,y_2)$ verwendet
werden kann.
Als zusätzliche (weiche) Bedingung ist also
die Summe der quadratischen 
              Euklidischen Distanzen zwischen verbundenen lokalen End- und Metroknoten 
zu minimieren.% \footnote{%
% Die Ausdrücke
% $(\sum_{1\leq i\leq m}\!\sqrt{|r_i|})<(\sum_{1\leq j\leq n}\!\sqrt{|s_j|})$ und
% $(\sum_{1\leq i\leq m}\!|r_i|)<(\sum_{1\leq j\leq n}\!|s_j|)$
% sind \"aquivalent.}

%\newpage
\paragraph{Repräsentation in ASP.}

In ASP repräsentieren wir Probleminstanzen durch Fakten folgender Form:
% \newpage
\vspace{-1mm}
\begin{verbatim}
 lNode(X,Y,L).   % lokaler Endknoten an Position (X,Y) erzeugt Last L
 mNode(X,Y,B,C). % möglicher Metroknoten an Position (X,Y) mit maximaler
                 % Kapazität B und Maximalanzahl C lokaler Endknoten
 metros(N).      % Anzahl N von zu verwendenden Metroknoten
 maxLoss(M).     % Schranke M abgetrennter Endknoten bei doppeltem Ausfall
\end{verbatim}

\noindent
Z.B.\ wird das in Abbildung~\ref{fig:metro} (links) dargestellte Netzplanungsproblem
durch folgende Fakten (mit positiven Integer-Koordinaten) beschrieben:
\vspace{-1mm}
\begin{verbatim}
 lNode(1,1,77).     lNode(1,2,53).     lNode(2,2,59).     lNode(4,1,47).
 mNode(1,3,160,3).  mNode(2,3,200,3).  mNode(3,1,130,2).  mNode(3,2,180,3).
 mNode(3,3,240,4).  mNode(4,2,240,4).  metros(3).         maxLoss(2).
\end{verbatim}

\noindent
Eine Lösung, d.h.\ eine Menge von Metroknoten und ihre Verbindungen
mit lokalen Endknoten, sodass die oben beschriebenen (harten) Bedingungen erf\"ullt sind,
wird durch Atome folgender Form repräsentiert:% 
\vspace{-1mm}
\begin{verbatim}
 metro(MX1,MY1) ... metro(MXn,MYn) % Metroknoten an (MXi,MYi)
 wires(LX1,LY1,MX1,MY1) ...        % Verbindungen von lokalen Endknoten an
 wires(LXm,LYm,MXm,MYm)            % (LXi,LYi) zu Metroknoten an (MXi,MYi)
\end{verbatim}
Die in Abbildung~\ref{fig:metro} (rechts) dargestellte Lösung wird z.B.\
durch die folgenden Atome innerhalb einer Antwortmenge beschrieben:
\vspace{-1mm}
\begin{verbatim}
 metro(2,3) metro(3,1) metro(3,2)
 wires(1,1,2,3) wires(1,1,3,1) wires(1,2,2,3) wires(1,2,3,2)
 wires(2,2,2,3) wires(2,2,3,2) wires(4,1,3,1) wires(4,1,3,2)
\end{verbatim}

% \paragraph{Beispiel.}

% Eine kleines Beispiel könnte wie folgt aussehen:
% \vspace{-1mm}
% \begin{verbatim}
% metros(3).
% maxLoss(2).
% lNode(1,1,77). lNode(2,2,59). lNode(1,2,53). lNode(4,1,47).
% mNode(2,4,211,5). mNode(3,2,609,2). mNode(1,4,714,6).
% mNode(4,3,17,2).  mNode(3,4,649,3). mNode(4,2,635,6).
% mNode(2,3,644,3). mNode(1,3,401,3). mNode(3,3,422,5).
% mNode(3,1,299,4).
% \end{verbatim}

% mit der dazugehörigen optimalen Antwortmenge:
% \begin{verbatim}
% $ gringo inst1.lp enc.lp | clasp
% > [...]
% > Answer: 4
% > metro(3,2) metro(1,3) metro(3,1)
% > wires(1,1,1,3) wires(1,1,3,1) wires(2,2,3,2) wires(2,2,3,1) 
% > wires(1,2,1,3) wires(1,2,3,1) wires(4,1,3,2) wires(4,1,3,1)
% > Optimization: 20
% > OPTIMUM FOUND
% \end{verbatim}

\paragraph{Framework.}

\noindent
In dem Archiv \href{http://www.cs.uni-potsdam.de/wv/lehre/11WS/11-Antwortprog/metro.zip}{\texttt{metro.zip}} finden Sie die Datei \texttt{metro.lp} und zwölf Beispielinstanzen.
Die Datei \texttt{metro.lp} ist von Ihnen mit Ihrem ASP-Encoding zu ergänzen,
sodass \emph{optimale} Antwortmengen zu Lösungen mit minimalen aufsummierten
Verbindungsstrecken korrespondieren.
Wenn Sie Ihr Encoding einreichen,
dann müssen die folgenden Zeilen in \texttt{metro.lp} enthalten sein:%
\footnote{Diese Anweisungen sorgen dafür, dass alle Atome, die nicht zur Repräsentation von Lösungen gehören, bei der Ausgabe von Antwortmengen ausgeblendet werden.}
\vspace{-1mm}
\begin{verbatim}
 #hide.
 #show metro/2.
 #show wires/4.
\end{verbatim}
\vspace{-1mm}
{\small\sf Sie m\"ussen Ihr Encoding in einer Datei mit dem Namen \texttt{metro.lp}
           einreichen.
           Neben \texttt{metro/2} und \texttt{wires/4} d\"urfen in der eingereichten Version keine weiteren
           Pr\"adikate in der Ausgabe enthalten    sein!}%

\paragraph{Formalitäten.}
%
Sie können die Praktikumsaufgabe in Gruppen von \textbf{bis zu zwei} Studenten gemeinsam bearbeiten.
Verschiedene Gruppen müssen verschiedene Lösungen einreichen.
Bei Plagiaten wird die Praktikumsaufgabe für alle beteiligten Gruppen als ``nicht bestanden'' gewertet.
Reichen Sie Ihr Encoding bitte bis zum \textbf{05.02.2012}
über die \href{http://yeti.haiti.cs.uni-potsdam.de/student/showPractical.do?practicalId=29}{YETI-Plattform} ein.
(Alle Gruppenmitglieder müssen bei YETI registriert werden!)
% Tragen Sie dort alle Ihre Gruppenmitglieder ein.
Achten Sie darauf, dass Sie Ihr Encoding in einer Datei mit dem Namen \code{metro.lp} einreichen,
wobei der Dateiname ausschließlich Kleinbuchstaben enthält.

\noindent
Neben den zwölf vorgegebenen Instanzen
testen wir Ihr Encoding auf weiteren Ihnen unbekannten Instanzen.
Auch auf diesen weiteren Instanzen muss Ihr Encoding \emph{korrekt} funktionieren,
d.h.\ (im Falle der Terminierung innerhalb weniger Minuten) \textbf{eine} optimale Lösung liefern,
damit die Praktikumsaufgabe als bestanden gilt.
Wenn Sie Ihr Encoding bei YETI hochladen, wird es dort (mit geringer zeitlicher Verz\"ogerung)
automatisch getestet.
Falls dabei Fehler auftreten,
d.h.\ eine optimale Lösung wurde nicht ermittelt,
k\"onnen Sie diese den Statusmeldungen von YETI entnehmen.
Zur Analyse der resultierenden Effizienz von Encodings werden wir ein Vergleichsencoding 
bei YETI hochladen, das Ihnen einen Anhalts\-punkt bzgl.\ der Performanz Ihres
eigenen Encodings gibt.

\noindent{\small\sf
\textbf{Bonus:}
Encodings, die eine ähnliche Performanz wie unser Vergleichsencoding erzielen,
werden mit bis zu 2 Bonuspunkten auf die Klausur honoriert.}

\paragraph{Empfehlungen und Hinweise:}\vspace{-2mm}
\begin{itemize}
\item Überlegen Sie sich, welche Bedingungen Sie zur Ermittlung (optimaler) L\"osungen
      testen wollen und definieren Sie geeignete Pr\"adikate, die diese Tests erm\"oglichen.
% eine allgemeine (geeignete) Repräsentation, um Pfade zu beschreiben, bevor Sie Ihre Idee in einem ASP-Encoding umsetzen.
\item Die resultierende Effizienz von Encodings kann nicht unabhängig von dem
      zugrunde liegende Lösungsverfahren ermittelt werden.
      Die folgenden Faustregeln können dennoch  (unabhängig von Lösungsverfahren) als universell betrachtet werden:
      \vspace{-3mm}
      \begin{enumerate}\itemsep-1.5pt
      \item Platzbedarf und Laufzeit korrespondieren zueinander.
            Wenn die gleichen Sachverhalte mit weniger (instanziierten) Regeln ausgedrückt werden können,
            dann sollte die Suche nach Antwortmengen davon profitieren.
            (Verwenden Sie möglichst Default-Negation anstelle von Auflistungen komplementärer Werte.)
      \item Regeln und Integritätsbedingungen sind umso stärker, je weniger (nicht statisch bestimmte)
            Vorbedingungen in ihren Körpern auftreten.
            (Vermeiden Sie für die Korrektheit unerhebliche Fallanalysen in Regelkörpern.)
      \item Die Suche profitiert i.d.R.\ davon, dass nicht-triviale inhärente Eigenschaften von (optimalen)
            Lösungen in``redundanten'' Bedingungen explizit gemacht werden, weil sie dann nicht
            erst auf Fallbasis über Suche ermittelt werden müssen.
      \end{enumerate}
      \vspace{-3mm}
\item Verwenden Sie die Kommandozeilenoption \texttt{--solution-recording} zur
      Effizienz\-steigerung bei der Optimierung von L\"osungen, z.B.\ wie in dem folgenden Aufruf:\newline
\begin{tabular}{@{}l}%
\texttt{[user@local \textasciitilde] clingo metro.lp inst1.lp --solution-recording 0}
\end{tabular}
\item Wenn Sie an einer Stelle nicht weiterkommen, können Sie sich gern an uns wenden.
      Wir versuchen alle Fragen bestmöglich zu beantworten.
      Fragen k\"onnen Sie entweder pers\"onlich an uns richten oder per Mail an:
      \href{mailto:asp1@cs.uni-potsdam.de}{\texttt{asp1@cs.uni-potsdam.de}}.
\item Fangen Sie bald mit der Bearbeitung der Aufgabe an, damit Ihnen die Zeit nicht davonläuft.
      (Sollten Sie trotzdem Schwierigkeiten mit der Einhaltung des Termins
       haben, dann wenden Sie sich bitte an uns, anstatt eine beliebige Lösung
       zu kopieren!)
% \item Behalten Sie Ihren Quelltext für sich.
%       Sie helfen Ihren Kommilitonen nicht, wenn Sie Ihren Quelltext weitergeben, da das Nachvollziehen eines fremden Encodings schwieriger ist als ein eigenes zu entwerfen.
\end{itemize}

\end{PraktikumsAufgabe}
\end{document}
