\documentclass[a4paper,12pt,ngerman]{article}

\usepackage[ngerman]{babel}
\usepackage[utf8]{inputenc}
%\usepackage{listings}
%\lstset{basicstyle=\ttfamily\small}
%\usepackage{verbatim}
%\usepackage{dingbat}
%\usepackage{graphics}
%\usepackage[dvipsnames,usenames]{color}
%\usepackage{amsmath,amssymb}
\usepackage{exercises}
\usepackage{../../../asp}
%\usepackage{url}
\usepackage{hyperref}

\sloppy
% \DeclareFontFamily{U}{bulb}{}
% \DeclareFontShape{U}{bulb}{m}{n}{<-> lightbulb10}{}
% \newcommand{\lightbulb}{{\usefont{U}{bulb}{m}{n}A}}

\begin{document}
\begin{PraktikumsAufgabe}{2}{(Sitzplan)}
%
% \vspace*{-6mm}
\paragraph{Problembeschreibung.}
%
%
Die Studenten eines Kurses überlegen sich, wie der Sitzplan bei der
Klausur aussehen müsste, damit möglichst viele von ihnen abschreiben können.
Zur Ermittlung aussichtsreicher Abschreibgelegenheiten 
schätzt jeder Student vorab seine Note.
Wenn ein Teilnehmer nicht zu nahe an der Klausuraufsicht sitzt,
kann er von einem besseren Studenten direkt vor oder neben sich abschreiben.
Um die Abschreibgelegenheiten objektiv zu optimieren, soll der
Sitzplan mittels Antwortmengenprogrammierung (ASP) erstellt werden.
Die ASP-Lösung ist allerdings auch für die Kursleitung praktisch,
um ihrerseits einen Sitzplan mit möglichst wenig Abschreibgelegenheiten zu gestalten.
%
\begin{figure}[h]
\begin{center}
\begin{tabular}[t]{l@{}p{5mm}@{}|@{}p{5mm}@{}|@{}p{5mm}@{}|@{}p{5mm}@{}|@{}l}
\cline{3-5}
& & & & & ~\textsf{1} 
\\\cline{3-5}
& & & & & ~\textsf{2} 
\\\cline{3-5}
& & & & & ~\textsf{3}
\\\cline{3-5}
& \multicolumn{1}{@{}c@{}}{} &
\multicolumn{1}{@{}c@{}}{\textsf{1}} & \multicolumn{1}{@{}c@{}}{\textsf{2}} & \multicolumn{1}{@{}c@{}}{\textsf{3}}
\\\multicolumn{6}{c}{Modus: Studenten}
\\\multicolumn{6}{c}{Beobachter: 1}
\end{tabular}
\begin{tabular}[t]{c|c}
Student & Note \\
\cline{1-2}
1 & 27 \\
2 & 14 \\
3 & 26 \\
4 & 39 \\
5 & 27 \\
6 & 34 \\
\end{tabular}
\hspace{3cm}
\begin{tabular}[t]{l@{}p{5mm}@{}|@{}p{5mm}@{}|@{}p{5mm}@{}|@{}p{5mm}@{}|@{}l}
\cline{3-5}
& & ~S & & ~3 & ~\textsf{1} 
\\\cline{3-5}
& & & ~2 & ~1 & ~\textsf{2}
\\\cline{3-5}
& & ~5 & ~4 & ~6& ~\textsf{3} 
\\\cline{3-5}
& \multicolumn{1}{@{}c@{}}{} &
\multicolumn{1}{@{}c@{}}{\textsf{1}} & \multicolumn{1}{@{}c@{}}{\textsf{2}} & \multicolumn{1}{@{}c@{}}{\textsf{3}} 
\end{tabular}
\end{center}
\vspace*{-4mm}
\caption{Ein Beispielszenario (links) und ein optimaler Sitzplan (rechts).\label{fig:plan}}
\end{figure}

\noindent
Ein beispielhaftes Szenario ist in Abbildung \ref{fig:plan} (links) dargestellt.
Neben dem Raum, in dem die Klausur geschrieben wird, sind die geschätzten Noten,
die Anzahl der für die Aufsicht zuständigen Personen und der Modus angegeben.
Die Aufgabe besteht darin, Studenten und Beobachter so zu platzieren,
dass möglichst viel (bzw.\ wenig) abgeschrieben werden kann,
wobei die folgenden formalen Kriterien angelegt werden:
\vspace{-3mm}
\begin{enumerate}\itemsep-3pt
  \item Verschiedene Beobachter müssen (aus Kostengründen)
        mindestens drei Plätze voneinander entfernt sitzen,
        wobei die Manhattan-Distanz (Summe der Koordinatendifferenzen)
        als Maß für den Abstand zwischen Plätzen
        verwendet wird. % \footnotemark[1]
  \item Ein Student, der horizontal, vertikal oder diagonal unmittelbar neben
        einem Beobachter sitzt,
        kann nicht abschreiben. % \footnote[1]{Die Distanzen sind in Manhattan-Metrik definiert.}
%        Die Position des Studenten, von dem abgeschrieben wird, ist dabei irrelevant.
%  \item Ein Student kann von einem besseren Studenten abschreiben, der direkt vor oder neben ihm sitzt.
%        Wie gut ein Student ist, wird dabei über die abgeschätzten Noten definiert.
        Andernfalls kann der Student von direkt vor oder neben sich platzierten
        besseren Teilnehmern (mit kleinerer Note) abschreiben.
  \item Im Studentenmodus soll die Anzahl der Abschreibgelegenheiten,
        ermittelt anhand von Paaren voneinander abschreibender Studenten,
        möglichst groß und im Kursleitungsmodus möglichst klein sein.
% Es soll die Anzahl der Abschreibmöglichkeiten maximiert werden und nicht die Anzahl der
%         Studenten die irgendwo abschreiben können. Das heißt, ein Student,
%         der links und rechts abschreiben kann, zählt doppelt.
%   \item Ob möglichst wenig oder viel abgeschrieben werden soll, bestimmt der Modus.
%         Im Studenten-Modus gilt es, die Abschreibmöglichkeiten zu maximieren und im Uni-Modus zu minimieren.
\end{enumerate}
\vspace{-3mm}
Eine optimale L\"osung (für Studenten) mit 6 Abschreibgelegenheiten ist in Abbildung~\ref{fig:plan} (rechts) dargestellt, wobei die Nummern für Studenten und $S$ für den Beobachter stehen.
Konkret kann bei der gezeigten Platzierung wie folgt abgeschrieben werden:
\vspace{-3mm}
\begin{itemize}\itemsep-3pt
\item Student $1$ von Student $2$ und $3$
\item Student $4$ von Student $2$, $5$ und $6$
\item Student $6$ von Student $1$
\end{itemize}

%\newpage
\paragraph{Repräsentation in ASP.}

In ASP repräsentieren wir Probleminstanzen durch Fakten
über den folgenden Prädikaten:
% \newpage
\vspace{-1mm}
\begin{verbatim}
 cols(C)            % Anzahl Sitze je Reihe
 rows(R)            % Anzahl der Sitzreihen
 mode(K)            % Modus (students - maximiere Abschreiben,
                    %        uni      - minimiere Abschreiben)
 numstudents(N)     % Anzahl der Studenten
 numsupervisors(M)  % Anzahl der Beobachter
 exp_grade(S,G)     % geschätzte Note G von Student S
\end{verbatim}
\vspace{-1mm}
Für jeden der von \texttt{1} bis \texttt{N} (aus \texttt{numstudents(N)})
fortlaufend nummerierten Studenten~\texttt{S} enhält eine Instanz genau
einen Fakt \texttt{exp\_grade(S,G)} mit der geschätzten Note~\texttt{G} von~\texttt{S}.

\noindent
Z.B.\ wird das Szenario in Abbildung~\ref{fig:plan} (links) % dargestellte
durch folgende Fakten beschrieben:
\vspace{-1mm}
\begin{verbatim}
 cols(3). rows(3). mode(students).
 numstudents(6). numsupervisors(1).
 exp_grade(1,27).  exp_grade(2,14).  exp_grade(3,26).
 exp_grade(4,39).  exp_grade(5,27).  exp_grade(6,34).
\end{verbatim}

\noindent
Eine Lösung, d.h.\ eine Platzierung von Studenten und Beobachtern,
% die Anzahl der Paare von Studenten (a,b) bei denen a von b abschreibt,
wird durch Atome der folgenden Form repräsentiert:%
\vspace{-1mm}
\begin{verbatim}
 at(student(1),X1,Y1) ... at(student(N),XN,YN)  % Studentenpositionen
 at(supervisor,X1,Y1) ... at(supervisor,XM,YM)  % Beobachterpositionen
\end{verbatim}
\vspace{-1mm}
Während die Nummern von Studenten in den Atomen enthalten sind,
werden Beobachter nicht einzeln benannt.
Außerdem müssen die Positionen in den Atomen paarweise ver\-schie\-den voneinander sein.

\noindent
Die in Abbildung~\ref{fig:plan} (rechts) dargestellte Lösung wird z.B.\
durch folgendes Atome innerhalb einer Antwortmenge beschrieben:
\vspace{-1mm}
\begin{verbatim}
 at(student(1),3,2)  at(student(2),2,2)  at(student(3),3,1)
 at(student(4),2,3)  at(student(5),1,3)  at(student(6),3,3)
 at(supervisor,1,1)
\end{verbatim}

\paragraph{Framework.}

\noindent
In dem Archiv \href{http://www.cs.uni-potsdam.de/wv/lehre/12WS/12-Antwortprog/plan.tar.gz}%TODO
{\texttt{plan.tar.gz}} finden Sie die Datei \texttt{plan.lp} und sieben Beispielinstanzen.
Die Datei \texttt{plan.lp} ist von Ihnen mit Ihrem ASP-Encoding zu ergänzen,
sodass \emph{optimale} Antwortmengen zu Lösungen mit je nach Modus
maximal oder minimal vielen Abschreibgelegenheiten
% "`Abschreiberpaaren"' 
korrespondieren.
Wenn Sie Ihr Encoding einreichen,
dann müssen die folgenden Zeilen in \texttt{plan.lp} enthalten sein:%
\footnote{Diese Anweisungen sorgen dafür, dass alle Atome, die nicht zur Repräsentation von Lösungen gehören, bei der Ausgabe von Antwortmengen ausgeblendet werden.}
\vspace{-1mm}
\begin{verbatim}
 #hide.
 #show at(student(_),_,_).
 #show at(supervisor,_,_).
\end{verbatim}
\vspace{-1mm}
{\small\sf Sie m\"ussen Ihr Encoding in einer Datei mit dem Namen \texttt{plan.lp}
           einreichen.
           Neben \texttt{at/3} d\"urfen in der eingereichten Version keine weiteren
           Pr\"adikate in der Ausgabe enthalten    sein!}%

\paragraph{Formalitäten.}
%
Sie können die Praktikumsaufgabe in Gruppen von \textbf{bis zu zwei} Studenten gemeinsam bearbeiten.
Verschiedene Gruppen müssen verschiedene Lösungen einreichen.
Bei Plagiaten wird die Praktikumsaufgabe für alle beteiligten Gruppen als ``nicht bestanden'' gewertet.
Reichen Sie Ihr Encoding bitte bis zum \textbf{31.12.2012}
über % die von der Moodle-Seite zur Vorlesung aus verlinkte 
\href{http://yeti.haiti.cs.uni-potsdam.de/student/showPractical.do?practicalId=38}{YETI} % -Plattform
ein.
(Alle Gruppenmitglieder müssen bei YETI einen Account haben und als 
 Gruppenmitglieder registriert sein!)
Achten Sie darauf, dass Sie Ihr Encoding in einer Datei mit dem Namen \texttt{plan.lp} einreichen,
wobei der Dateiname ausschließlich Kleinbuchstaben enthält.

\noindent
Neben den sieben vorgegebenen Instanzen
testen wir Ihr Encoding auf weiteren Ihnen unbekannten Instanzen.
Auch auf diesen weiteren Instanzen muss Ihr Encoding \emph{korrekt} funktionieren,
d.h.\ (im Falle der Terminierung innerhalb weniger Minuten) \textbf{eine} optimale Lösung liefern,
damit die Praktikumsaufgabe als bestanden gilt.
Wenn Sie Ihr Encoding bei YETI hochladen, wird es dort automatisch getestet
(mit geringer zeitlicher Verz\"ogerung).
Falls dabei Fehler auftreten,
% d.h.\ eine optimale Lösung wurde nicht ermittelt,
k\"onnen Sie dies den Statusmeldungen von YETI entnehmen.
Bitte korrigieren Sie eventuelle Fehler umgehend selbständig oder
(bei Schwierigkeiten) in Rücksprache mit uns.

\noindent{\small\sf
\textbf{Bonus:} % TODO
Zur Analyse der resultierenden Effizienz von Encodings werden wir ein Vergleichsencoding 
bei YETI hochladen, das Ihnen einen Anhalts\-punkt bzgl.\ der Performanz Ihres
eigenen Encodings gibt.
Encodings, die eine ähnliche Performanz wie unser Vergleichsencoding erzielen,
werden mit bis zu 2 Bonuspunkten auf die Klausur honoriert.}

\paragraph{Empfehlungen und Hinweise:}\vspace{-2mm}
\begin{itemize}
\item Überlegen Sie sich, welche Bedingungen Sie zur Ermittlung (optimaler) L\"osungen
      testen wollen und definieren Sie geeignete Pr\"adikate, die diese Tests erm\"oglichen.
% eine allgemeine (geeignete) Repräsentation, um Pfade zu beschreiben, bevor Sie Ihre Idee in einem ASP-Encoding umsetzen.
\item Die resultierende Effizienz von Encodings kann nicht unabhängig von dem
      zugrunde liegende Lösungsverfahren ermittelt werden.
      Die folgenden Faustregeln können dennoch  (unabhängig von Lösungsverfahren) als universell betrachtet werden:
      \vspace{-2mm}
      \begin{enumerate}\itemsep-1pt
      \item Platzbedarf und Laufzeit korrespondieren zueinander.
            Wenn die gleichen Sachverhalte mit weniger (instanziierten) Regeln ausgedrückt werden können,
            dann sollte die Suche nach Antwortmengen davon profitieren.
            (Verwenden Sie möglichst Default-Negation anstelle von Auflistungen komplementärer Werte.)
      \item Regeln und Integritätsbedingungen sind umso stärker, je weniger (nicht statisch bestimmte)
            Vorbedingungen in ihren Körpern auftreten.
            (Vermeiden Sie für die Korrektheit unerhebliche Fallanalysen in Regelkörpern.)
      \item Die Suche profitiert i.d.R.\ davon, dass nicht-triviale inhärente Eigenschaften von (optimalen)
            Lösungen in``redundanten'' Bedingungen explizit gemacht werden, weil sie dann nicht
            erst auf Fallbasis über Suche ermittelt werden müssen.
      \end{enumerate}
%      \vspace{-3mm}
% \item Verwenden Sie die Kommandozeilenoption \texttt{--solution-recording} zur
%       Effizienz\-steigerung bei der Optimierung von L\"osungen, z.B.\ wie in dem folgenden Aufruf:\newline
% \begin{tabular}{@{}l}%
% \texttt{[user@local \textasciitilde] clingo plan.lp test01.lp --solution-recording 0}
% \end{tabular}
\item Wenn Sie an einer Stelle nicht weiterkommen, können Sie sich gern an uns wenden.
      Wir versuchen alle Fragen bestmöglich zu beantworten.
      Fragen und Bemerkungen k\"onnen Sie pers\"onlich an uns richten,
      in \href{http://moodle.cs.uni-potsdam.de/course/view.php?id=39}{Moodle} (Forum/Wiki) stellen oder per Mail an
      \href{mailto:asp1@cs.uni-potsdam.de}{\texttt{asp1@cs.uni-potsdam.de}} senden.
\item Fangen Sie bald mit der Bearbeitung der Aufgabe an, damit Ihnen die Zeit nicht davonläuft.
      (Sollten Sie trotzdem Schwierigkeiten mit der Einhaltung des Termins
       haben, dann wenden Sie sich bitte an uns, anstatt eine beliebige Lösung
       zu kopieren!)
% \item Behalten Sie Ihren Quelltext für sich.
%       Sie helfen Ihren Kommilitonen nicht, wenn Sie Ihren Quelltext weitergeben, da das Nachvollziehen eines fremden Encodings schwieriger ist als ein eigenes zu entwerfen.
\end{itemize}

\end{PraktikumsAufgabe}
\end{document}
