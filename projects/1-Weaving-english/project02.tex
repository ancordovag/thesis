\documentclass[a4paper,12pt]{article}

%\usepackage[ngerman]{babel}
\usepackage[utf8]{inputenc}
%\usepackage{listings}
%\lstset{basicstyle=\ttfamily\small}
%\usepackage{verbatim}
%\usepackage{dingbat}
\usepackage{graphics}
\usepackage[dvipsnames,usenames,table]{xcolor}
\usepackage{amsmath,amssymb}
\usepackage{exercises}
\usepackage{../../asp}
\usepackage{url}
\usepackage{hyperref}
\usepackage{tikz}

\sloppy
\DeclareFontFamily{U}{bulb}{}
\DeclareFontShape{U}{bulb}{m}{n}{<-> lightbulb10}{}
\newcommand{\lightbulb}{{\usefont{U}{bulb}{m}{n}A}}

\newcommand*\circled[1]{\hspace{0.6mm}\tikz[baseline=(char.base)]{
    \node[shape=circle,draw,fill,inner sep=0.2mm] (char) {\color{White}\footnotesize{\textsf{\textbf{#1}}}};}}

\newcommand*\ccircle[1]{\tikz[baseline={([yshift=-8pt]current bounding box.north)}]{\draw[black,fill=#1] (0,0) circle (1ex);}}

\begin{document}
\begin{PraktikumsAufgabe}{2}{(Weaving)}
\vspace*{-4mm}
\paragraph{Problem Description.}
%
%
The task of this project is to solve a logic puzzle using ASP.
The objective of the game is to paint a board one row or column at a time, 
so that the final board matches the goal colors.

At each step the player paints the cells in exactly one column or row, 
and the game is finished after each column and row has been painted exactly once.
Painting a column or row replaces the color of each cell with the color chosen by the player.
To win the game the color of every cell has to match its goal color,
and the cells with no goal color may be colored arbitrarily.

%%%%%
%rr %
%g b%
%r g%
%%%%%
%
\begin{figure}[h]
\begin{center}
\begin{tabular}{c@{~}|c|c|c|l}
\cline{2-4}
\textsf{1} & \ccircle{red} & \ccircle{red} &
\\\cline{2-4}
\textsf{2} & \ccircle{green} & & \ccircle{blue}
\\\cline{2-4}
\textsf{3} & \ccircle{red} & \ccircle{green} & &
\\\cline{2-4}
\multicolumn{1}{@{}c@{}}{} &
\multicolumn{1}{@{}c@{}}{\textsf{1}} & \multicolumn{1}{@{}c@{}}{\textsf{2}} & \multicolumn{1}{@{}c@{}}{\textsf{3}} 
\end{tabular}
\begin{tabular}{c@{~}|c|c|c|l}
\cline{2-4}
\textsf{1} & \ccircle{red} & \ccircle{red} &
\\\cline{2-4}
\textsf{2} & \ccircle{green} & & \ccircle{blue}
\\\cline{2-4}
\textcolor{red}{\textsf{3}} & {\cellcolor{red!50}}\ccircle{red} & {\cellcolor{red!50}}\ccircle{green} & {\cellcolor{red!50}} &
\\\cline{2-4}
\multicolumn{1}{@{}c@{}}{} &
\multicolumn{1}{@{}c@{}}{\textsf{1}} & \multicolumn{1}{@{}c@{}}{\textsf{2}} & \multicolumn{1}{@{}c@{}}{\textsf{3}} 
\end{tabular}
\begin{tabular}{c@{~}|c|c|c|l}
\cline{2-4}
\textsf{1} & {\cellcolor{red!50}}\ccircle{red} & \ccircle{red} &
\\\cline{2-4}
\textsf{2} & {\cellcolor{red!50}}\ccircle{green} & & \ccircle{blue}
\\\cline{2-4}
\textsf{3} & {\cellcolor{red!50}}\ccircle{red} & {\cellcolor{red!50}}\ccircle{green} & {\cellcolor{red!50}} &
\\\cline{2-4}
\multicolumn{1}{@{}c@{}}{} &
\multicolumn{1}{@{}c@{}}{\textcolor{red}{\textsf{1}}} & \multicolumn{1}{@{}c@{}}{\textsf{2}} & \multicolumn{1}{@{}c@{}}{\textsf{3}} 
\end{tabular}
\begin{tabular}{c@{~}|c|c|c|l}
\cline{2-4}
\textsf{1} & {\cellcolor{red!50}}\ccircle{red} & {\cellcolor{green!50}}\ccircle{red} &
\\\cline{2-4}
\textsf{2} & {\cellcolor{red!50}}\ccircle{green} & {\cellcolor{green!50}} & \ccircle{blue}
\\\cline{2-4}
\textsf{3} & {\cellcolor{red!50}}\ccircle{red} & {\cellcolor{green!50}}\ccircle{green} & {\cellcolor{red!50}} &
\\\cline{2-4}
\multicolumn{1}{@{}c@{}}{} &
\multicolumn{1}{@{}c@{}}{\textsf{1}} & \multicolumn{1}{@{}c@{}}{\textcolor{green}{\textsf{2}}} & \multicolumn{1}{@{}c@{}}{\textsf{3}} 
\end{tabular}
\begin{tabular}{c@{~}|c|c|c|l}
\cline{2-4}
\textsf{1} & {\cellcolor{red!50}}\ccircle{red} & {\cellcolor{green!50}}\ccircle{red} &
\\\cline{2-4}
\textcolor{green}{\textsf{2}} & {\cellcolor{green!50}}\ccircle{green} & {\cellcolor{green!50}} & {\cellcolor{green!50}}\ccircle{blue}
\\\cline{2-4}
\textsf{3} & {\cellcolor{red!50}}\ccircle{red} & {\cellcolor{green!50}}\ccircle{green} & {\cellcolor{red!50}} &
\\\cline{2-4}
\multicolumn{1}{@{}c@{}}{} &
\multicolumn{1}{@{}c@{}}{\textsf{1}} & \multicolumn{1}{@{}c@{}}{\textsf{2}} & \multicolumn{1}{@{}c@{}}{\textsf{3}} 
\end{tabular}
\begin{tabular}{c@{~}|c|c|c|l}
\cline{2-4}
\textcolor{red}{\textsf{1}} & {\cellcolor{red!50}}\ccircle{red} & {\cellcolor{red!50}}\ccircle{red} & {\cellcolor{red!50}}
\\\cline{2-4}
\textsf{2} & {\cellcolor{green!50}}\ccircle{green} & {\cellcolor{green!50}} & {\cellcolor{green!50}}\ccircle{blue}
\\\cline{2-4}
\textsf{3} & {\cellcolor{red!50}}\ccircle{red} & {\cellcolor{green!50}}\ccircle{green} & {\cellcolor{red!50}} &
\\\cline{2-4}
\multicolumn{1}{@{}c@{}}{} &
\multicolumn{1}{@{}c@{}}{\textsf{1}} & \multicolumn{1}{@{}c@{}}{\textsf{2}} & \multicolumn{1}{@{}c@{}}{\textsf{3}} 
\end{tabular}
\begin{tabular}{c@{~}|c|c|c|l}
\cline{2-4}
\textsf{1} & {\cellcolor{red!50}}\ccircle{red} & {\cellcolor{red!50}}\ccircle{red} & {\cellcolor{blue!50}}
\\\cline{2-4}
\textsf{2} & {\cellcolor{green!50}}\ccircle{green} & {\cellcolor{green!50}} & {\cellcolor{blue!50}}\ccircle{blue}
\\\cline{2-4}
\textsf{3} & {\cellcolor{red!50}}\ccircle{red} & {\cellcolor{green!50}}\ccircle{green} & {\cellcolor{blue!50}} &
\\\cline{2-4}
\multicolumn{1}{@{}c@{}}{} &
\multicolumn{1}{@{}c@{}}{\textsf{1}} & \multicolumn{1}{@{}c@{}}{\textsf{2}} & \multicolumn{1}{@{}c@{}}{\textcolor{blue}{\textsf{3}}} 
\end{tabular}

\end{center}
\vspace*{-6mm}
\caption{Example with a $3\times 3$ board and a step by step solution.\label{fig:weaving}}
\end{figure}

\noindent
One example is shown in Figure~\ref{fig:weaving}.
The top left board shows the initial configuration with six goal colors,
and the following boards represent the steps of one possible solution.
The column or row just painted is marked with a colored number on the side or bottom of the table.
The last table shows the final state with all cells matching the goal colors.

%\newpage
\paragraph{Representation in ASP.}
The dimensions of the board,
the number of steps,
the available colors
and the goal colors are represented by facts of the following predicates:

% \newpage
\vspace{-1.5mm}
\begin{verbatim}
dim(D).                % D is the dimension of the board
steps(S).              % S is the number of steps
color(C).              % C is an available color
goal(X,Y,C).           % Cell [X,Y] has goal color C
\end{verbatim}
\vspace{-1mm}
Additionally, the predicate \code{painted/4} is used to fix some of the painting steps.
The first parameter represents whether a row (\verb+x+) or column (\verb+y+) is painted,
the second gives the row or column number,
the third the color, and the last represents at which step of the game the action takes place
(and the first step has number 1).
\begin{verbatim}
painted(x,N,C,S).      % row    N must be painted with color C at step S
painted(y,N,C,S).      % column N must be painted with color C at step S
\end{verbatim}
\newpage
For instance, the example shown in Figure~\ref{fig:weaving} (top left) is
represented by the following facts:%
\vspace{-1.5mm}
\begin{verbatim}
dim(3). steps(6). color(red). color(green). color(blue).
goal(1,1,red).      goal(1,2,red).
goal(2,1,green).                         goal(2,3,blue).
goal(3,1,red).      goal(3,2,green).
painted(x,3,red,1). painted(y,1,red,2).
\end{verbatim}
\vspace{-1mm}
A solution -- i.e. the ordered list of painted columns and rows -- is represented by facts of predicate \texttt{paint/4}:
\vspace{-1.5mm}
\begin{verbatim}
paint(x,N,C,S).        % paint row    N with color C at step S
paint(y,N,C,S).        % paint column N with color C at step S
\end{verbatim}
\vspace{-1mm}
The parameters of \code{paint/4} are identical to those of \code{painted/4}.
The solution in Figure~\ref{fig:weaving} is represented by the following atoms:
\vspace{-1.5mm}
\begin{verbatim}
paint(x,3,red,1)    paint(y,1,red,2)  paint(y,2,green,3)
paint(x,2,green,4)  paint(x,1,red,5)  paint(y,3,blue,6)
\end{verbatim}
\vspace{-1mm}
Note that the first two steps are identical to the ones given by \code{painted/4}.


\paragraph{Framework.}

\noindent
In the
%\href{http://www.cs.uni-potsdam.de/wv/lehre/13WS/??-Antwortprog/weaving.tar.gz}
{\texttt{weaving.zip}}
archive at Moodle you will find five example instances.
You have to submit a file named \texttt{weaving.lp} that contains the following line 
(and no more \texttt{\#show} statements)
so that in the output only the atoms of predicate \texttt{paint/4} appear:
%\footnote{These instructions hide all atoms that are not part of the solution representation.}%
\vspace{-1.5mm}
\begin{verbatim}
#show paint/4.
\end{verbatim}
% \begin{tabular}{l}
% \code{\#hide.}\\
% \code{\#show paint/2.}\\
% \end{tabular}
\vspace{-1mm}
% \noindent
%{\small\sf The file containing your encoding has to be called \texttt{sudoku.lp}.
%           Apart from \code{cell/3} no additional predicates must be included in the output!}

\noindent

\paragraph{Formalities.}
%
You can work on the solution alone or in groups of two people.
Different groups have to submit different solutions, 
in case of plagiarism all groups involved will fail the project.
Please submit your encoding by \textbf{Thursday 27th, 2014} via
\href{https://yeti.haiti.cs.uni-potsdam.de}{YETI}
(All group members have to create a YETI account!).
Be sure to submit your encoding in a file named \code{weaving.lp}
containing only lowercase letters.

\noindent
We will test your encoding with the five provided instances as well as additional instances.
Your encoding has to find the correct solutions for all instances.
This will be tested automatically by YETI after you uploaded the encoding (with a slight delay).
If your solution is not correct then YETI will display an error message.
Please correct any errors that occur on your own or contact us if you get stuck.

\paragraph{Tips:}
\begin{itemize}
\item Commands to find all stable models look as follows:%
\vspace{-1.5mm}%
\begin{verbatim}
$  gringo-4 weaving.lp example.lp | clasp3 0
\end{verbatim}
\vspace{-1mm}
\item If you are stuck you can contact us. We will do out best to answer all your questions.
      You can send us questions and remarks via 
      \href{http://moodle.cs.uni-potsdam.de/course/view.php?id=39}{Moodle} (forum/wiki)
      or send them via mail to 
      \href{mailto:asp@lists.cs.uni-potsdam.de}{\texttt{asp@lists.cs.uni-potsdam.de}}.
\item Start as soon as possible to avoid running out of time.
      (However, if you still realize that you have problems making it before the deadline,
       please contact us instead of copying another solution).
% \item Behalten Sie Ihren Quelltext für sich.
%       Sie helfen Ihren Kommilitonen nicht, wenn Sie Ihren Quelltext weitergeben, da das Nachvollziehen eines fremden Encodings schwieriger ist als ein eigenes zu entwerfen.
\end{itemize}

\end{PraktikumsAufgabe}
\end{document}

